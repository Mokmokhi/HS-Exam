\documentclass[12pt]{article}
\usepackage{ctex}
\usepackage[english]{babel}
\usepackage{blindtext}
\usepackage{nameref}
\usepackage{fancyhdr}
\usepackage{amsmath,amssymb,amsthm}
\usepackage{graphicx,float}
\usepackage{physics}
\usepackage{pgfplots}
\usepackage[a4paper, total={6in, 9in}]{geometry}

\pagestyle{fancy}
\fancyhf{}
\fancyhf[HL]{中四 黃隸珊下考1920 翻譯試題}
\fancyhf[CF]{\thepage}

\newcommand{\innerprod}[2]{\langle{#1},{#2}\rangle}
\newcommand{\id}{\mathtt{id}}

\begin{document}
    \begin{enumerate}
        \item 化簡$\dfrac{(ab^{-1})^{64}}{a^{-25}}$並以正指數表示答案。
        \item 因式分解:\begin{enumerate}
            \item $a^4+2a^2b^2+b^4$.
            \item $a^4+a^2b^2+b^4$.
        \end{enumerate}
        \item 假設一直角三角形的邊長為$(6-x)$,$(13-x)$及$(14-x)$。\begin{enumerate}
            \item 求$x$的值。
            \item 求三角形的面積。
        \end{enumerate}
        \item 在第一學期,中四學員的男女比例為$3:2$。若男生的數量增加$14\%$而女生數量下降$18\%$,求中四學員總人數的百分改變。
        \item 在極坐標系統中,$O$為原點。設$A$和$B$的極坐標分別爲$(7,22.5^\circ)$及$(3,112.5^\circ)$。設$C$令$O$為$AC$的中點。\begin{enumerate}
            \item 試描述$OB$與$AC$的幾何關係,並加以解釋。
            \item 求$\triangle ABC$的面積。
        \end{enumerate}
        \item 某兩個數字的和與差分別爲$18$和$12$。試在不求出該兩個數字的情況下,求他們的積。
        \item 已知$h(x)=x^2+kx+1$。若$h(-2)=h(5)$,\begin{enumerate}
            \item 求$k$的值。
            \item 設$g(x)=h(x-1)-5$。考慮$g(x)=0$的根的和,程偉認爲$g(x)$的對稱軸為$x=5$。你是否認同?試加以解釋。
        \end{enumerate}
        \item 中五學生共180人,以下圓形圖分別顯示參與課外活動的男女人數。已知其中一半的女生只參與一種課外活動,而男生平均每人參與1.5種課外活動。\begin{enumerate}
            \item 求$x$。
            \item 隨機選擇一名中五學生,選出參與3種課外活動的男生的概率相等於選出參與3種課外活動的男生的概率。求中五女生人數。
        \end{enumerate}
        \item 正圓錐體的底半徑和高度分別爲$3r$和$3r+15$。\begin{enumerate}
            \item 以$r$表示圓錐體的體積。
            \item 若圓錐體的體積為$54\pi$,求$r$的值。
        \end{enumerate}
        \item 已知$h(x)$其中一部分為常數,另一部分正變於$x$。假設$h(721)=1443$及$h(831)=1663$。\begin{enumerate}
            \item 求$h(x)$。
            \item 解方程$h(x)=x^2$。以根號表示答案。
            \item 若$g(x)=x^2-h(x)$,利用(b)的結果,求$g(x)$的極小值。
        \end{enumerate}
        \item $\alpha$和$\beta$為$x^2-2kx-(3-k)=0$的根。若$\frac{1}{\alpha}+\frac{1}{\beta}=\frac{1}{2}$,求\begin{enumerate}
            \item $k$。
            \item $\beta^2-2\alpha$。
        \end{enumerate}
        \item $A$點的坐標為$(2,3)$。$B$點為$A$點逆時針旋轉$270^\circ$的結果。同時,先將$A$點沿y軸反射,再向下移動1個單位並向右移動1個單位到達$C$點。\begin{enumerate}
            \item 寫下$B$和$C$的坐標。
            \item 求通過$B$和$C$的直綫方程。
            \item 設$L$通過$A$點並垂直於$BC$,求$L$的直綫方程。由此,或以其他方法,求$\triangle ABC$的面積。
        \end{enumerate}
        \item 設$f(x)=2x^3-5x^2-x+4k$,其中$k$為常數。當$f(x)$被$x-k$除時,其商為$g(x)$且餘數為$k$。\begin{enumerate}
            \item 求$k$的值。
            \item 假設$k$為正整數。解方程$f(x)=k$。
            \item 設$0<k<1$。比利認爲$y=g(x)$的圖像與直綫$y=4$相交於兩個相異點。你同意嗎?試加以解釋。
        \end{enumerate}
        \item \begin{enumerate}
            \item 以$a+bi$的形式表示$\frac{5}{\sqrt{2}+\sqrt{3}i}$,其中$a,b$為實數。
            \item 設$p,q$為實數使得$\frac{5}{\sqrt{2}+\sqrt{3}i}$為$x^2-px-q=0$的其中一個解,求$p$與$q$。
        \end{enumerate}
        \item \begin{enumerate}
            \item 若直綫$y=ax+b$通過$(-1,3)$且斜率為$3$,求$a$和$b$。
            \item 設$\log_{\sqrt{2}}y=a\log_2{x}+b$,其中$a,b$為於(a)的值。試以$y=Ax^k$的形式表述$x$和$y$的關係,其中$A$和$k$為常數。
        \end{enumerate}
        \item 設$f(x)=2x^2-4x+11$。\begin{enumerate}
            \item 解$f(x)=0$。如有需要,以$a+bi$的形式表達答案。
            \item 利用配方法,寫出頂點坐標及對稱軸。
        \end{enumerate}
        \item \begin{enumerate}
            \item 解方程$2u^2-u-1=0$。
            \item 利用(a)的結果,解以下方程:\begin{enumerate}
                \item $2(x-1)^2-x=0$.
                \item $2^{2x+1}-2^x=1$.
                \item $4(\log{x})^2-\log{x^2}-\log{100}=0$.
            \end{enumerate}
        \end{enumerate}
        \item 假設$\alpha$為以下二次方程組的共解:\begin{align*}
            \begin{cases}
                x^2-4x\sin{\theta}-2=0\\
                x^2-4x\cos{\theta}+2=0
            \end{cases}
        \end{align*}\begin{enumerate}
            \item 證明$\alpha=\dfrac{1}{\cos{\theta}-\sin{\theta}}$
            \item 由此,證明$\sin^2{\theta}=\dfrac{1}{4}$。
            \item 若$\sin{\theta}>0$,求$y=x^2-4x\sin{\theta}-2$的極小值及對應的$x$的值。
        \end{enumerate}
    \end{enumerate}
\end{document}